% 2018.01.08 Modified
% 2015.07.10 Modified
%
% mthesis.tex
%
%\documentclass[12pt]{jarticle} % Japanese
\documentclass[12pt]{article} % English
% if there are problems in the above regarding fonts, use this
% \documentclass[UTF8]{ctexart}

\usepackage[utf8]{inputenc}
%\usepackage{utf}
\usepackage{naist-jmthesis} %Japanese
%\usepackage{naist-mthesis} %English

\usepackage{graphicx}

%
% Page style
%
\pagestyle{final}       % Camera-Ready
%\pagestyle{draft}      % Draft
%
%
%\lang{Japanese} % Japanese
%\lang{English} % English
%
% Student Number
%
\studentnumber{1811098}
%
% 修士論文 か 課題研究 かの選択
%
\doctitle{\mastersthesis}       % 修士論文
%\doctitle{\mastersreport}      % 課題研究
%
% 取得予定の修士号は 修士(工学) か 修士(理学) か ?
%
\major{\engineering}    % 工学
%\major{\science}       % 理学
%
% 所属プログラムはどこか ?
%
\program{\ise}          % 情報理工学プログラム / Program of Information Science and Engineering
%\program{\cb}           % 情報生命科学プログラム / Program of Computational Biology
%\program{\icps}         % 知能社会創成科学プログラム / Program of Computational Biology
%\program{\ds}           % データサイエンス / Program of Data Science
%
% 日本語題目 (in LaTeX)
%
\title{ソースコードの類似性に基づいたテストコード\\自動推薦ツールSuiteRec}
%
% 日本語題目 (in plain text)
%
%   注: (in LaTeX)と同じ場合は指定する必要なし。
%       この情報は修士論文/課題研究には現れませんが、管理のために必要です。
%
\ptitle{ソースコードの類似性に基づいたテストコード\\自動推薦ツールSuiteRec}
%
% 英語題目 (in LaTeX)
%
\etitle{Automatic Test Suite Recommendation System based on Code Clone Detection}
%
% 英語題目 (in plain text)
%
%   注: (in LaTeX)と同じ場合は指定する必要なし。
%       この情報は修士論文/課題研究には現れませんが、管理のために必要です。
%
\eptitle{Automatic Test Suite Recommendation System based on Code Clone Detection}
%
% 日本語氏名 (in LaTeX)
%   (姓と名の間に空白を入れて下さい)
%
\author{倉地 亮介}
%
% 日本語氏名 (in plain text)
%
%   注: (in LaTeX)と同じ場合は指定する必要なし。
%       この情報は修士論文/課題研究には現れませんが、管理のために必要です。
%
\pauthor{}
%
% 欧文氏名 (in LaTeX)
%   (first name, last name の順に記入し、先頭文字のみを大文字にする。)
%
\eauthor{Ryosuke Kurachi}
% 別の例: \eauthor{Kurt G\"{o}del}
%
%
% 欧文氏名 (in plain text)
%
%   注: (in LaTeX)と同じ場合は指定する必要なし。
%       この情報は修士論文/課題研究には現れませんが、管理のために必要です。
%
\epauthor{}
% 別の例: \peauthor{Kurt Goedel}
%
%
% 論文提出年月日
%
\esyear{2020}
\jsyear{令和2}
\smonth{3}
\sday{20}
%
% 専攻の選択
%
%\department{\infproc}  % 情報処理学
%\department{\infsys}    % 情報システム学
%\department{\bioinf}   % 情報生命科学
%\department{\infsci}    % 情報科学
%
%
% 審査委員(日本語)
%   (姓と名、名と称号の間に空白を入れて下さい)
%
%5人以上の場合,5人目以降は\addcmembers を使って宣言する。
%最大で合わせて8人まで宣言可能。
%主指導教員、副指導教員を明記する。両指導教員以外は委員。
%学外審査委員は所属を明記する。しかし、理由により所属を記載できない場合は任意で構いません。
%
% 4人の場合
\cmembers{飯田 元 教授}{(主指導教員,情報科学領域)}
         {井上 美智子 教授}{(副指導教員,情報科学領域)}
         {市川 昊平 准教授}{(副指導教員,情報科学領域)}
         {高橋 慧智 助教}{(副指導教員,情報科学領域)}
%
% 3人の場合
%\cmembers{○○ ○○ 教授}{(主指導教員)}
%         {○○ ○○ 教授}{(副指導教員)}
%         {○○ ○○ 准教授}{(副指導教員)}
%         {}{}
%
% 2人の場合
%\cmembers{○○ ○○ 教授}{(主指導教員)}
%         {○○ ○○ 教授}{(副指導教員)}
%          {}{}
%          {}{}
%
% 5人目の宣言
\addcmembers{崔 恩瀞 助教}{(京都工芸繊維大学)}
            {}{}
            {}{}
            {}{}

% 5〜6人目の宣言
%\addcmembers{55 55 准教授}{(□□大学)}
%            {66 66 准教授}{(□□大学)}
%            {}{}
%            {}{}
%
% 5〜7人目の宣言
%\addcmembers{55 55 准教授}{(□□大学)}
%            {66 66 准教授}{(□□大学)}
%            {77 77 准教授}{(□□大学)}
%            {}{}
%
% 5〜8人目の宣言
%\addcmembers{55 55 准教授}{(□□大学)}
%            {66 66 准教授}{(□□大学)}
%            {77 77 准教授}{(□□大学)}
%            {88 88 准教授}{(□□大学)}
%
%
% 審査委員(英語)
%     (first name, last name の順に記入し、先頭文字のみを大文字にする。
%       first name と last name の間に空白、
%       last name と 称号の間にカンマと空白を入れて下さい。)
%
% 5人以上の場合,5人目以降は\eaddcmembers を使って宣言する
% Supervisor, Co-supervisor, and Member must be basically specified.
% Exceptionally, the affiliation of the co-supervisor does not have to be
% specified when it is unavailable for some reason.
% 4人の場合
\ecmembers{Professor XXX XXX}{(Supervisor, Division of Information Science)}
          {Professor XXX XXX}{(Co-supervisor, Division of Information Science)}
          {Associate Professor XXX XXX}{(Co-supervisor, Division of Information Science)}
          {Associate Professor XXX XXX}{(YY University)}
%
% 3人の場合
%\ecmembers{Professor XXX XXX}{(Supervisor)}
%          {Professor XXX XXX}{(Co-supervisor)}
%          {Associate Professor XXX XXX}{(Co-supervisor)}
%          {}{}
%
% 2人の場合
% \ecmembers{Professor XXX XXX}{(Supervisor)}
%           {Professor XXX XXX}{(Co-supervisor)}
%           {}{}
%           {}{}
%
% 5人目の宣言
%\eaddcmembers{Professor 55 55}{(YY University)}
%            {}{}
%            {}{}
%            {}{}
%
% 5〜6人目の宣言
%\eaddcmembers{Professor 55 55}{(YY University)}
%             {Professor 66 66}{(YY University)}
%             {}{}
%             {}{}
%
% 5〜7人目の宣言
%\eaddcmembers{Professor 55 55}{(YY University)}
%             {Professor 66 66}{(YY University)}
%             {Professor 77 77}{(YY University)}
%             {}{}
%
% 5〜8人目の宣言
%\eaddcmembers{Professor 55 55}{(YY University)}
%             {Professor 66 66}{(YY University)}
%             {Professor 77 77}{(YY University)}
%             {Professor 88 88}{(YY University)}
%
%
%
% キーワード5〜6個 (in LaTeX)
%
\keywords{コードクローン検出,推薦システム,ソフトウェアテスト,テストスメル,単体テスト}
%
% キーワード5〜6個 (in plain text)
%
%   注: (in LaTeX)と同じ場合は記入する必要なし。
%       この情報は修士論文/課題研究には現れませんが、管理のために必要です。
%
\pkeywords{コードクローン検出,推薦システム,ソフトウェアテスト,テストスメル,単体テスト}
%
% 5 or 6 Keywords (in LaTeX)
%
\ekeywords{clone detection, recommendation system, software testing, test smell, unit test}
%
% 5 or 6 Keywords (in plain text)
%
%   注: (in LaTeX)と同じ場合は記入する必要なし。
%       この情報は修士論文/課題研究には現れませんが、管理のために必要です。
%
\epkeywords{clone detection, recommendation system, software testing, test smell, unit test}
%
% 内容梗概 (in LaTeX)
%
%   注: 行の先頭が\\で始まらないようにすること。
%
\abstract{
ソフトウェアの品質確保の要と言えるソフトウェアテストを支援することは,重要である.これまでにテスト工程を支援するために,様々な自動生成技術が提案されてきた.しかし,既存技術によって自動生成されたテストコードは,テスト対象コードの作成経緯や意図に基づいて生成されていないので,開発者の保守作業を困難にさせる.この課題の解決方法として,既存テストの再利用が有効であると考えられる.本研究では,オープンソースソフトウェアに存在する品質が高いテストコードを推薦するツール{\sf SuiteRec}を提案する.推薦手法のアイディアは,類似するソースコード間でテストコードを再利用することである.開発者からの入力コード片に対して類似コード片を検出し,その類似コード片に対するテストスイートを推薦する.さらに,テストコードの良くない実装を表す指標であるテストスメルを開発者に提示し,より品質の高いテストスイートを推薦できるように推薦順位を並び替える.{\sf SuiteRec}の有用性を評価した被験者実験では,{\sf SuiteRec}を使用した場合とそうでない場合で,テスト作成をどの程度支援できるかを定量的および定性的に評価した.その結果,{\sf SuiteRec}を利用した場合,(1) 条件分岐が多いプログラムのテストコードを作成する際にコードカバレッジの向上に効果的であること,(2) 作成したテストコードはテストスメルの数が少なく品質が高いこと,(3) 開発者はテストコード作成作業を容易だと認識し,自身で作成したテストコードに自信が持てることが分かった.また,{\sf SuiteRec}は開発者が参考にしたいテストスイートを上位に推薦できることを確認した.
}
%
% 内容梗概 (in plain text)
%
%   注: (in LaTeX)と同じ場合は記入する必要なし。
%       この情報は修士論文/課題研究には現れませんが、管理のために必要です。
%       改行する箇所には空白行を入れる。
%       行の先頭が\\で始まらないようにすること。
%
\pabstract{
人類がこの地上に現われて以来、piの計算には多くの関心が払われてきた。

本論文では、太陽と月を利用してpiを低速に計算するための
画期的なアルゴリズムを与える。

ここには内容梗概を書く。ここには内容梗概を書く。ここには内容梗概を書く。
ここには内容梗概を書く。ここには内容梗概を書く。ここには内容梗概を書く。
ここには内容梗概を書く。ここには内容梗概を書く。ここには内容梗概を書く。
ここには内容梗概を書く。ここには内容梗概を書く。ここには内容梗概を書く。
ここには内容梗概を書く。ここには内容梗概を書く。ここには内容梗概を書く。

ここには内容梗概を書く。ここには内容梗概を書く。ここには内容梗概を書く。
ここには内容梗概を書く。ここには内容梗概を書く。ここには内容梗概を書く。
ここには内容梗概を書く。ここには内容梗概を書く。ここには内容梗概を書く。
ここには内容梗概を書く。ここには内容梗概を書く。ここには内容梗概を書く。
ここには内容梗概を書く。ここには内容梗概を書く。ここには内容梗概を書く。
}
%
% Abstract (in LaTeX)
%
%  注:  行の先頭が\\で始まらないようにすること。
%
\eabstract{
The calculation of $\pi$ has been paid much attention since human beings
appeared on the earth.

This thesis presents novel low-speed algorithms to calculate
$\pi$ utilizing the sun and the moon.

This is a sample abstract. This is a sample abstract. 
This is a sample abstract. This is a sample abstract. 
This is a sample abstract. This is a sample abstract. 
This is a sample abstract. This is a sample abstract. 
This is a sample abstract. This is a sample abstract. 

This is a sample abstract. This is a sample abstract. 
This is a sample abstract. This is a sample abstract. 
This is a sample abstract. This is a sample abstract. 
This is a sample abstract. This is a sample abstract. 
This is a sample abstract. This is a sample abstract. 
}
%
% Abstract (in plain text)
%
%   注: (in LaTeX)と同じ場合は記入する必要なし。
%       この情報は修士論文/課題研究には現れませんが、管理のために必要です。
%       改行する箇所には空白行を入れる。
%       行の先頭が\\で始まらないようにすること。
%
\epabstract{
The calculation of pi has been paid much attention since human beings
appeared on the earth.

This thesis presents novel low-speed algorithms to calculate
pi utilizing the sun and the moon.

This is a sample abstract. This is a sample abstract. 
This is a sample abstract. This is a sample abstract. 
This is a sample abstract. This is a sample abstract. 
This is a sample abstract. This is a sample abstract. 
This is a sample abstract. This is a sample abstract. 

This is a sample abstract. This is a sample abstract. 
This is a sample abstract. This is a sample abstract. 
This is a sample abstract. This is a sample abstract. 
This is a sample abstract. This is a sample abstract. 
This is a sample abstract. This is a sample abstract. 
}
%%%%%%%%%%%%%%%%%%%%%%%%% document starts here %%%%%%%%%%%%%%%%%%%%%%%%%%%%
\begin{document}
%
% 表紙 および アブストラクト
%
\titlepage
\cmemberspage
\firstabstract
\secondabstract
%
% 目次
%
\toc
\newpage
\listoffigures
%\newpage
\listoftables
%
% これ以降本文
%
\newpage
\section{はじめに}
\pagenumbering{arabic}

はじめに はじめに はじめに はじめに はじめに はじめに はじめに はじめに 
はじめに はじめに はじめに はじめに はじめに はじめに はじめに はじめに 
はじめに はじめに はじめに はじめに はじめに はじめに はじめに はじめに 

はじめに はじめに はじめに はじめに はじめに はじめに はじめに はじめに 
はじめに はじめに はじめに はじめに はじめに はじめに はじめに はじめに 
はじめに はじめに はじめに はじめに はじめに はじめに はじめに はじめに 

\ref{kako}節では、過去における研究について述べ、
\ref{kadai}章では、現状と今後の課題について述べる。
また、付録\ref{omake1}におまけその1を添付する。


\subsection{過去における研究}
\label{kako}


\begin{figure*}[t]
 \centering
% \includegraphics[width=0.8\textwidth,keepaspectratio,clip]{fig/cnn.eps}
 \caption{Convolutional Neural Network (CNN)}
 \label{fig:CNN}
\end{figure*}

過去における研究としては\cite{alex_nips12}などがある。

過去における研究 過去における研究 過去における研究 
過去における研究 過去における研究 過去における研究 過去における研究 
過去における研究 過去における研究 過去における研究 過去における研究 

過去における研究 過去における研究 過去における研究 過去における研究 
過去における研究 過去における研究 過去における研究 過去における研究 
過去における研究 過去における研究 過去における研究 過去における研究 
過去における研究 過去における研究 過去における研究 過去における研究 
過去における研究 過去における研究 過去における研究 過去における研究 

過去における研究 過去における研究 過去における研究 過去における研究 
過去における研究 過去における研究 過去における研究 過去における研究 
過去における研究 過去における研究 過去における研究 過去における研究 
過去における研究 過去における研究 過去における研究 過去における研究 
過去における研究 過去における研究 過去における研究 過去における研究 

\subsection{研究の目的と意義}

研究の目的と意義 研究の目的と意義 研究の目的と意義 研究の目的と意義 
研究の目的と意義 研究の目的と意義 研究の目的と意義 研究の目的と意義 
研究の目的と意義 研究の目的と意義 研究の目的と意義 研究の目的と意義 
研究の目的と意義 研究の目的と意義 研究の目的と意義 研究の目的と意義 

研究の目的と意義 研究の目的と意義 研究の目的と意義 研究の目的と意義 
研究の目的と意義 研究の目的と意義 研究の目的と意義 研究の目的と意義 
研究の目的と意義 研究の目的と意義 研究の目的と意義 研究の目的と意義 
研究の目的と意義 研究の目的と意義 研究の目的と意義 研究の目的と意義 

研究の目的と意義 研究の目的と意義 研究の目的と意義 研究の目的と意義 
研究の目的と意義 研究の目的と意義 研究の目的と意義 研究の目的と意義 
研究の目的と意義 研究の目的と意義 研究の目的と意義 研究の目的と意義 
研究の目的と意義 研究の目的と意義 研究の目的と意義 研究の目的と意義 

\begin{figure}
\centerline{ここに図を書く}
\caption{これは図の例}
\end{figure}

\begin{table}
\centerline{ここに表を書く}
\caption{これは表の例}
\end{table}

研究の目的と意義 研究の目的と意義 研究の目的と意義 研究の目的と意義 
研究の目的と意義 研究の目的と意義 研究の目的と意義 研究の目的と意義 
研究の目的と意義 研究の目的と意義 研究の目的と意義 研究の目的と意義 
研究の目的と意義 研究の目的と意義 研究の目的と意義 研究の目的と意義 

研究の目的と意義 研究の目的と意義 研究の目的と意義 研究の目的と意義 
研究の目的と意義 研究の目的と意義 研究の目的と意義 研究の目的と意義 
研究の目的と意義 研究の目的と意義 研究の目的と意義 研究の目的と意義 
研究の目的と意義 研究の目的と意義 研究の目的と意義 研究の目的と意義 

研究の目的と意義 研究の目的と意義 研究の目的と意義 研究の目的と意義 
研究の目的と意義 研究の目的と意義 研究の目的と意義 研究の目的と意義 
研究の目的と意義 研究の目的と意義 研究の目的と意義 研究の目的と意義 
研究の目的と意義 研究の目的と意義 研究の目的と意義 研究の目的と意義 

研究の目的と意義 研究の目的と意義 研究の目的と意義 研究の目的と意義 
研究の目的と意義 研究の目的と意義 研究の目的と意義 研究の目的と意義 
研究の目的と意義 研究の目的と意義 研究の目的と意義 研究の目的と意義 
研究の目的と意義 研究の目的と意義 研究の目的と意義 研究の目的と意義 

研究の目的と意義 研究の目的と意義 研究の目的と意義 研究の目的と意義 
研究の目的と意義 研究の目的と意義 研究の目的と意義 研究の目的と意義 
研究の目的と意義 研究の目的と意義 研究の目的と意義 研究の目的と意義 
研究の目的と意義 研究の目的と意義 研究の目的と意義 研究の目的と意義 

研究の目的と意義 研究の目的と意義 研究の目的と意義 研究の目的と意義 
研究の目的と意義 研究の目的と意義 研究の目的と意義 研究の目的と意義 
研究の目的と意義 研究の目的と意義 研究の目的と意義 研究の目的と意義 
研究の目的と意義 研究の目的と意義 研究の目的と意義 研究の目的と意義 

研究の目的と意義 研究の目的と意義 研究の目的と意義 研究の目的と意義 
研究の目的と意義 研究の目的と意義 研究の目的と意義 研究の目的と意義 
研究の目的と意義 研究の目的と意義 研究の目的と意義 研究の目的と意義 
研究の目的と意義 研究の目的と意義 研究の目的と意義 研究の目的と意義 

研究の目的と意義 研究の目的と意義 研究の目的と意義 研究の目的と意義 
研究の目的と意義 研究の目的と意義 研究の目的と意義 研究の目的と意義 
研究の目的と意義 研究の目的と意義 研究の目的と意義 研究の目的と意義 
研究の目的と意義 研究の目的と意義 研究の目的と意義 研究の目的と意義 

研究の目的と意義 研究の目的と意義 研究の目的と意義 研究の目的と意義 
研究の目的と意義 研究の目的と意義 研究の目的と意義 研究の目的と意義 
研究の目的と意義 研究の目的と意義 研究の目的と意義 研究の目的と意義 
研究の目的と意義 研究の目的と意義 研究の目的と意義 研究の目的と意義 

研究の目的と意義研究の目的と意義研究の目的と意義研究の目的と意義 
研究の目的と意義研究の目的と意義研究の目的と意義研究の目的と意義 
研究の目的と意義研究の目的と意義研究の目的と意義研究の目的と意義 
研究の目的と意義研究の目的と意義研究の目的と意義研究の目的と意義 

研究の目的と意義研究の目的と意義研究の目的と意義研究の目的と意義 
研究の目的と意義研究の目的と意義研究の目的と意義研究の目的と意義 
研究の目的と意義研究の目的と意義研究の目的と意義研究の目的と意義 
研究の目的と意義研究の目的と意義研究の目的と意義研究の目的と意義 

研究の目的と意義研究の目的と意義研究の目的と意義研究の目的と意義 
研究の目的と意義研究の目的と意義研究の目的と意義研究の目的と意義 
研究の目的と意義研究の目的と意義研究の目的と意義研究の目的と意義 
研究の目的と意義研究の目的と意義研究の目的と意義研究の目的と意義 


\newpage

This page is written in English. This page is written in English. 
This page is written in English. This page is written in English. 
This page is written in English. This page is written in English. 
This page is written in English. This page is written in English. 

This page is written in English. This page is written in English. 
This page is written in English. This page is written in English. 
This page is written in English. This page is written in English. 
This page is written in English. This page is written in English. 

This page is written in English. This page is written in English. 
This page is written in English. This page is written in English. 
This page is written in English. This page is written in English. 
This page is written in English. This page is written in English. 

This page is written in English. This page is written in English. 
This page is written in English. This page is written in English. 
This page is written in English. This page is written in English. 
This page is written in English. This page is written in English. 

This page is written in English. This page is written in English. 
This page is written in English. This page is written in English. 
This page is written in English. This page is written in English. 
This page is written in English. This page is written in English. 

This page is written in English. This page is written in English. 
This page is written in English. This page is written in English. 
This page is written in English. This page is written in English. 
This page is written in English. This page is written in English. 

This page is written in English. This page is written in English. 
This page is written in English. This page is written in English. 
This page is written in English. This page is written in English. 
This page is written in English. This page is written in English. 
This page is written in English. This page is written in English. 
This page is written in English. This page is written in English. 
This page is written in English. This page is written in English. 
This page is written in English. This page is written in English. 

This page is written in English. This page is written in English. 
This page is written in English. This page is written in English. 
This page is written in English. This page is written in English. 
This page is written in English. This page is written in English. 
This page is written in English. This page is written in English. 
This page is written in English. This page is written in English. 
This page is written in English. This page is written in English. 
This page is written in English. This page is written in English. 

This page is written in English. This page is written in English. 
This page is written in English. This page is written in English. 
This page is written in English. This page is written in English. 
This page is written in English. This page is written in English. 
This page is written in English. This page is written in English. 
This page is written in English. This page is written in English. 
This page is written in English. This page is written in English. 
This page is written in English. This page is written in English. 


\newpage
\section{現状と今後の課題}
\label{kadai}

現状と今後の課題 現状と今後の課題 現状と今後の課題 現状と今後の課題 
現状と今後の課題 現状と今後の課題 現状と今後の課題 現状と今後の課題 
現状と今後の課題 現状と今後の課題 現状と今後の課題 現状と今後の課題 
現状と今後の課題 現状と今後の課題 現状と今後の課題 現状と今後の課題 

現状と今後の課題 現状と今後の課題 現状と今後の課題 現状と今後の課題 
現状と今後の課題 現状と今後の課題 現状と今後の課題 現状と今後の課題 
現状と今後の課題 現状と今後の課題 現状と今後の課題 現状と今後の課題 
現状と今後の課題 現状と今後の課題 現状と今後の課題 現状と今後の課題 

現状と今後の課題 現状と今後の課題 現状と今後の課題 現状と今後の課題 
現状と今後の課題 現状と今後の課題 現状と今後の課題 現状と今後の課題 
現状と今後の課題 現状と今後の課題 現状と今後の課題 現状と今後の課題 
現状と今後の課題 現状と今後の課題 現状と今後の課題 現状と今後の課題 

%
% 謝辞
%
\acknowledgements

Thank you. Thank you.
%
% 参考文献
% ここでは \reference を使って、自分でリストを作るか、BibTeX を使って
% リストをつくって下さい。この例では BibTeX を作るような形式になってい
% ます。
%
\newpage
% \reference
\bibliographystyle{plain}
\bibliography{mthesis}
%
% 付録
%
\appendix

\section{おまけその1}
\label{omake1}

これはおまけです。これはおまけです。これはおまけです。これはおまけです。
これはおまけです。これはおまけです。これはおまけです。これはおまけです。
これはおまけです。これはおまけです。これはおまけです。これはおまけです。
これはおまけです。これはおまけです。これはおまけです。これはおまけです。

\begin{figure}
\centerline{これはおまけの図です。}
\caption{おまけの図}
\end{figure}


\section{おまけその2}

これもおまけです。これもおまけです。これもおまけです。これもおまけです。
これもおまけです。これもおまけです。これもおまけです。これもおまけです。
これもおまけです。これもおまけです。これもおまけです。これもおまけです。
これもおまけです。これもおまけです。これもおまけです。これもおまけです。

\end{document}

